\documentclass{article}

\usepackage{amsmath}
\usepackage{amssymb}
\usepackage{amsthm}
\usepackage{enumerate}
\usepackage{bbm}
\usepackage{lipsum}
\usepackage{fancyhdr}
\usepackage{tikz-cd} 

\newtheorem{theorem}{Theorem}[section] 
\newtheorem{proposition}{Proposition}[section] 
\newtheorem{definition}{Definition}[section] 
\newtheorem{lemma}{Lemma}[section] 
\newtheorem{notation}{Notation}[section] 
\newtheorem{remark}{Remark}[section] 
\newtheorem{corollary}{Corollary}[section] 
\newtheorem{terminology}{Terminology}[section] 
\newtheorem{example}{Example}[section] 
\numberwithin{equation}{section}

\DeclareMathOperator{\diam}{diam}
\DeclareMathOperator{\rk}{rk}
\DeclareMathOperator{\rank}{rank}
\DeclareMathOperator{\Hom}{Hom}
\DeclareMathOperator{\Dom}{Dom}
\DeclareMathOperator{\grad}{grad}
\DeclareMathOperator{\Span}{Span}
\DeclareMathOperator{\interior}{int}
\DeclareMathOperator{\ind}{ind}
\DeclareMathOperator{\supp}{supp}
\DeclareMathOperator{\sgn}{sgn}
\DeclareMathOperator{\ob}{ob}
\DeclareMathOperator{\Spec}{Spec}
\DeclareMathOperator{\PreSh}{PreSh}
\DeclareMathOperator{\Fun}{Fun}
\DeclareMathOperator{\Ker}{Ker}


\title{Representation Theory 1 V4A3 Exercise Sheet 1}
\author{So Murata}
\date{2024/2025 Winter Semester - Uni Bonn}

\begin{document}
\maketitle

\section*{Exercise 1}

\subsection*{1.1 $\mathcal{C}^\infty$ is a locally ringed space}

Let us define a set such that 
\begin{equation*}
\mathfrak{m}_{X,p}=\{(U,f)\:|\: f\in\mathcal{C}^\infty(U), f(p)=0\}.
\end{equation*}

Since $f:U\to\mathbb{R}$ is a smooth map and $f(p)\not=0$, then the quotient ${\frac 1 f}$ is smooth at $p$, we derive that $\mathfrak{m}_{X,p}$ is a unique maximal ideal of $\mathcal{O}_{X,p}$.

\subsection*{1.2 A smooth map induces a map of locally ringed space.}

For any $x\in g^{-1}(V)$, we have $g(x)\in V$ by the definition, we obtain that

\begin{align*}
f|_V\circ g|_{g^{-1}(V)} = (f\circ g)|_{g^{-1}(V)}.
\end{align*}
This shows that $g^{\#}$ is a natural transformation. Furthermore, for any $f:U\to\mathbb{R}$ such that $f(p) = 0$, we obtain for any $q\in g^{-1}(\{p\})$, $f\circ g(q)=0$. Thus the image of maximal ideal under $g$ is contained a maximal idea. This proves the claim.

\subsection*{1.3}

Composition of $g$ and each component of a chart is smooth thus a composition of $g$ and any chart is smooth. This concludes that $g$ is smooth.

\section*{Exercise 2}

\subsection*{2.1}

Let us define the product topology on $M\times N$, since it is a direct product of finitely many topological space, it coincides with the box topology. And atlases of it is defined as the product of two atlases, ie. for any atlas 
\begin{equation*}
\mathcal{A}_{M\times N} = \{h_M\times h_N:U_M\times U_N\to V_M\times V_N\:|\: h_M\in\mathcal{A}_M,h_N\in\mathcal{A}_N\}
\end{equation*}

for some atlases $\mathcal{A}_M$ and $\mathcal{A}_N$. By the construction of the product topology, such set is indeed an atlas. \\

Let $h:U\to V$ and $(h_M,h_N):U_M\times U_N\to V_M\times V_N$ be charts of $M,M\times N$, respectively. Then
\begin{equation*}
h\circ \pi_1\circ (h_M,h_N)^{-1} = h\circ \pi_1\circ (h_M^{-1},h_N^{-1}) = h\circ h_M^{-1}.
\end{equation*}

By the assumption, this is smooth, therefore $\pi_1$ is a smooth map.\\

\par Given a map of smooth manifolds $f:M'\to M\times N$. $f$ is smooth if and only if 
\begin{equation*}
(h_M,h_N)\circ f\circ h'^{-1}
\end{equation*}
is smooth for any charts. By the elementary tool from analysis we know that this means, the function is coordinate-wise smooth thus the above function is smooth if and only if 
\begin{equation*}
h_M\circ\pi_1\circ f\circ h'^{-1},h_N\circ\pi_2\circ f\circ h'^{-1}
\end{equation*}
are smooth. \\

\par Since $\pi_1,\pi_2$ are both smooth, any topology on $M\times N$ with these criterions would contain the product topology on it. Since identity map of $M\times N$ is smooth, we obtain that such $M\times N$ is unique. 

\subsection*{2.2}

Using the coordinate tangent space we obtain that 
\begin{equation*}
((h_M,h_N),(v_M,v_n)) \mapsto (h_M,v_M)\times(h_N,v_N).
\end{equation*}

By the construction of atlases in $M\times N$, this map is a bijection. 

\section*{Exercise 3}

\subsection*{3.1}

Let $\alpha,\beta:I\to G$ be smooth curves such that $\alpha(0)=\beta(0)=e$ and $\alpha,\beta$ are representations of equivalent classes $X,Y$ in $T_{(e,e)}^{\mathbf{Geo}}(G\times G)$. Since derivatives are linear maps we obtain
\begin{equation*}
d\mu_{(e,e)}(X,Y) = d\mu_{(e,e)}(X,e)+d\mu_{(e,e)}(e,Y).  
\end{equation*}

By using Geometric tangent space we get
\begin{equation*}
d\mu_{(e,e)}(X,e) = {\frac d {dt}}|_0\mu(\alpha(t),e) = \alpha'(0) = X.
\end{equation*}

By applying this to $Y$ we obtain

\begin{equation*}
d\mu_{(e,e)}(X,Y) = X+Y.
\end{equation*}

\subsection*{3.2}

Trivially we have $\mu(e,e) =e$ and the derivative is surjective as it is proven in the previous problem. \\
Let $X=h, Y=0$ for any $h\in T_{g,g^{-1}}G$ then this is a surjection thus full rank. We use the implicit function theorem and conclude there is $\iota$ such that
\begin{equation*}
\mu(a,\iota(a))=e
\end{equation*}
around a neighborhood of $e$. And this is smooth as $\mu$ is smooth. Let $\alpha:I\to G$ be a smooth curve.
\begin{equation*}
\mu(\alpha(t),\iota(\alpha(t)) = e.
\end{equation*}

Therefore
\begin{equation*}
d_{(e,e)}\mu(X,\iota(X)) = 0 = \left(\alpha'(0),\alpha'(0)\iota'(e)\right)\begin{pmatrix}1\\1\end{pmatrix} = X+d_e\iota X = 0.
\end{equation*}

\subsection*{3.3}

Let $g\in G$ and consider $\mu(g,g^{-1})=e$. Using similar arguments in the 3.1 we obtain that

\begin{equation*}
d_{g,g^{-1}}\mu(X,Y) = d_{g,g^{-1}}(X,g^{-1})+d_{g,g^{-1}}(g,Y).
\end{equation*}

We pick $\alpha,\beta:I\to G$ to be such that
\begin{equation*}
\alpha(0) = g, \beta(0) = g^{-1}.
\end{equation*} 

Then 
\begin{equation*}
d_{g,g^{-1}}\mu(X,Y) = Xd(g^{-1})+d(g)Y.
\end{equation*}

Let $X=hg, Y=0$ for any $h\in T_{g,g^{-1}}G$ then this is a surjection thus full rank. We can use the implicit function theorem and by the uniqueness of inverse, we conclude that $G$ is a Lie group.

\end{document}